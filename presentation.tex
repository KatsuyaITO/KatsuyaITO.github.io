\documentclass[dvipdfmx,22pt,notheorems]{beamer}
%%%% 和文用 %%%%%
\usepackage{bxdpx-beamer}
\usepackage{pxjahyper}
\usepackage{minijs}%和文用
\renewcommand{\kanjifamilydefault}{\gtdefault}%和文用

%%%% スライドの見た目 %%%%%
\usetheme{Madrid}
\usefonttheme{professionalfonts}
\setbeamertemplate{frametitle}[default][center]
\setbeamertemplate{navigation symbols}{}
\setbeamercovered{transparent}%好みに応じてどうぞ)
\setbeamertemplate{footline}[page number]
\setbeamerfont{footline}{size=\normalsize,series=\bfseries}
\setbeamercolor{footline}{fg=black,bg=black}
%%%%

%%%% 定義環境 %%%%%
\usepackage{amsmath,amssymb}
\usepackage{amsthm}
\theoremstyle{definition}
\newtheorem{theorem}{定理}
\newtheorem{definition}{定義}
\newtheorem{proposition}{命題}
\newtheorem{lemma}{補題}
\newtheorem{corollary}{系}
\newtheorem{conjecture}{予想}
\newtheorem*{remark}{Remark}
\renewcommand{\proofname}{}
%%%%%%%%%

%%%%% フォント基本設定 %%%%%
\usepackage[T1]{fontenc}%8bit フォント
\usepackage{textcomp}%欧文フォントの追加
\usepackage[utf8]{inputenc}%文字コードをUTF-8
\usepackage{otf}%otfパッケージ
\usepackage{bm}%数式太字
\setbeamerfont{normal text}{size=\Large} 
%%%%%%%%%%
 
\title[略タイトル]{グレブナー基底と代数多様体}%[略タイトル]{タイトル}
\author[]{伊藤克哉@WeLoveBuddha}%[略名前]{名前}
\date{\today}%日付
\begin{document}

\begin{frame}[plain]\frametitle{}
\titlepage %表紙
\end{frame}

\begin{frame}\frametitle{目次}
\tableofcontents %目次
\end{frame}

\section{6学期はじめの私}
\begin{frame}
	\frametitle{6学期はじめの私}
		\begin{itemize}
			\item \Large 谷村大先輩と荒田大先輩の煽りを受けてこの教室に。
			\item \Large グレブナー基底に興味はあったが知識は0。
			\item \Large 代数幾何の知識は僅かにあった。
		\end{itemize}
	
\end{frame}

\begin{frame}
	\frametitle{6学期はじめの目標}
		\begin{itemize}
			\item \Large グレブナー基底と代数幾何の関わりを見たい。
			\item \Large 具体的な代数多様体を愛でたい。
		\end{itemize}
	
\end{frame}
\section{結果}
\begin{frame}
	\frametitle{今学期読んだ本}
	Cox, David A, Little, John, O'Shea, Donal著のIdeals, Varieties, and Algorithms
	(邦題:グレブナー基底と代数多様体)を主に読んだ.その他この本で参照されている論文などを読んだ.グレブナー基底についてはそんなに書いてないけど代数幾何  (古典論)を勉強するにはとても良い本だった.
\end{frame}
\begin{frame}
	\frametitle{グレブナー基底とは}
	$R=K[X_1,\cdots,X_n]$という多項式と$I \subset R$イデアルを考える.\\
	$g \in I$であることの判定法を与えるのがグレブナー基底.
	\begin{theorem}
	$\{ f_1, \cdots , f_n \}$を$I$のグレブナー基底とする\\
	$g \in I \Leftrightarrow g$を $\{f_1, \cdots , f_n \}$で割った余りが$0$\\
	\end{theorem}
\end{frame}
\begin{frame}
	\frametitle{Computer Algebra System}
	コンピューター上で数式を処理するためのソフトをComputer Algebra Systemという.グレブナー基底を計算するのにも必要.
	\begin{table}[htb]
  \begin{tabular}{|l|c|l|} \hline
    名前 & ライセンス & 特徴  \\ \hline 
    Mathematica & 学生\$140 & 代表格 \\
    Maxima & Free & Mathematicaに引けをとらない\\
	Axiom & Free & 代数もできる\\
	Macaulay2 & Free & 代数幾何に強い\\
	Sympy & Free & Pythonベースで書きやすい\\ \hline
  \end{tabular}
\end{table}
\end{frame}
\begin{frame}
	\frametitle{6学期中の私}
		\begin{itemize}
			\item \large 初め:Cox.et.alの「グレブナー基底と代数多様体」を読む\pause
			\item \large その後:Cox「素イデアルを判定するアルゴリズムがあるよ」\pause
			\item \large $\to$ 論文をあさりPythonで実装をしようとする。\pause
			\item \large ar$\chi$iv 「グレブナー基底を使ってコホモロジーが計算できるよ」\pause
			\item \large$\to$ 論文をあさりMacaulay2で実装しようとする。\pause
			\item \Huge 実装は辛い。
		\end{itemize}
\end{frame}
\subsection{結果1}
\begin{frame}
	\frametitle{結果1:グレブナー基底に関する知見を得た}
		\begin{itemize}
		\item \Large グレブナー基底のキーとなる性質は最初に述べた
		\item \Large 可換環:準素分解・素イデアルの判定
		\item \Large 代数幾何:次元の計算・コホモロジーの計算
		\item \Large 暗号理論:多変数公開鍵暗号
		\end{itemize}
\end{frame}
\subsection{結果2}
\begin{frame}
	\frametitle{結果2:準素分解のアルゴリズムをしった}
		\begin{itemize}
		\item Gianni, P.; Trager, B.; Zacharias, G.: Gröbner Bases and
Primary Decomposition of Polynomial Ideals. J. Symb. Comp.
6, 149–167 (1988).
		\item Shimoyama, T.; Yokoyama, K.: Localization and Primary
Decomposition of Polynomial ideals. J. Symb. Comp. 22,
247–277 (1996).
		\end{itemize}
		などのアルゴリズムを勉強した.実装は出来なかった.
\end{frame}
\section{まとめ}
\begin{frame}
	\frametitle{まとめ}
			\begin{itemize}
			\item \Huge グレブナー基底はすごい.
			\item \Huge 代数幾何は楽しい.
		\end{itemize}
\end{frame}
\section{今後}
\begin{frame}
	\frametitle{今後}
		\begin{itemize}
			\item \Large 論文はたくさん書かれているけど実装はされてない。
			\item \Large SymPyを完成させたい。
			\item \Large 実装力がほしい。
		\end{itemize}
\end{frame}

\begin{frame}
	\frametitle{おまけ}
	\begin{block}{問題}
	$x^2+y^2=16,x^3+y^3=44$という条件下で$x+y$の値を求めよ.
	\end{block}
	という問題をグレブナー基底を使って解いてみる.\\
	$I=\langle x^2+y^2=16,x^3+y^3=44,s-(x+y) \rangle$の生成元を求める\par
	その生成元で$s$だけで生成されているものを見れば$x+y$の値が求まる.
\end{frame}

\begin{frame}[containsverbatim]
\begin{verbatim}
i1 : R=RR[x,y,s]
o1 = R
o1 : PolynomialRing 
\end{verbatim}
\begin{verbatim}
i2 : I=ideal(x^2+y^2-16,x^3+y^3-44,s-(x+y))
             2    2        3    3
o2 = ideal (x  + y  - 16, x  + y  - 44, - x - y + s)
o2 : Ideal of R
\end{verbatim}
\begin{verbatim}
i3 : gens gb I
o3 = | x+y-s y2-ys+.5s2-8 s3-48s+88 |
             1       3
o3 : Matrix R  <--- R
\end{verbatim}
より$s^3-48s+88 = (s-2)(s^2+2s-44)=0$を解くと$x+y=2$
\end{frame}
\end{document}


